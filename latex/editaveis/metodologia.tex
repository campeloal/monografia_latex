\chapter[Metodologia]{Metodologia}

\section{Levantamento Bibliográfico}

Sabendo do interesse, dentro da área de computação gráfica, especificamente no desenvolvimento de  \textit{shaders}, primeiramente foi feito um levantamento bibliográfico, a fim de avaliar a disponibilidade de material para fomentar o tema de trabalho de pesquisa e também analisar o que já foi desenvolvido na área. Feito isso, tomando-se como base o que já foi publicado e desenvolvido, foram definidas as possíveis contribuições, em que (como foi dito na seção 1 Introdução) viu-se que a limitação de desempenho e o desenvolvimento para \textit{mobile} são áreas a serem exploradas.  

\section{Configuração do Ambiente}
	
Assim, primeiramente foram feitas as configurações dos ambientes de trabalho, em que -  como citado na Seção 2.10  -  para desenvolver na plataforma \textit{Android} é necessário instalar o \textit{Android SDK} e o \textit{plugin} ADT, este último a fim de poder desenvolver na IDE \textit{Eclipse}. A biblioteca gráfica para sistemas embarcados \textit{OpenGL ES} já é oferecida pela plataforma \textit{Android}. Para computador, foi necessário instalar as bibliotecas \textit{GLUT}, \textit{GLEW} e por fim, a biblioteca gráfica \textit{OpenGL}.

\section{Equipamentos Utilizados}

O celular utilizado foi o \textit{Nexus} 4, no qual é o quarto  \textit{smartphone} da  \textit{Google}, projetado e fabricado pela \textit{LG Electronics}.  Ele possui o processador \textit{Snapdragon S4 Pro} de 1,512 GHz \textit{quad-core}, GPU \textit{Adreno} 320 e 2 GB de memória RAM. O computador utilizado foi o da linha \textit{Alienware} M14x fabricado pela \textit{Dell}l, no qual possui processador \textit{Intel Core} i7 de 2,3 GHz, GPU \textit{NVIDIA GeForce} GTX de 2 GB e 8 GB de memória RAM. 

\section{Definição do Tema}

Primeiramente foi analisado se era factível estender o tema também para a plataforma \textit{Android} - tanto no que diz respeito ao prazo quanto em relação também ao conhecimento já possuído. Então avaliou-se o o nível de dificuldade de implementação de um  \textit{shader} para plataforma \textit{Android} (principalmente por não possuir experiência prévia com desenvolvimento \textit{mobile}), desenvolvendo um \textit{shader} simples aplicado num octaedro. Também desenvolveu-se o mesmo \textit{shader} para computador, analisando as as diferenças de implementação entre eles.  

Feito isto, também foi realizado um levantamento de ferramentas de otimização gráfica tanto para \textit{Android} como para computador, no qual escolheram-se as ferramentas \textit{Adreno} e \textit{gDEBugger}, respectivemente. Essas ferramentas são utilizadas a fim de coletar medições quanto ao número de quadros por segundo de cada programa, utilizando um \textit{shader} específico, aplicado num objeto tridimensional com n número de polígonos. 

A fim de facilitar a implementação dos \textit{shaders}, também foi realizado um levantamento de ferramentas para o desenvolvimento de \textit{shaders}, em que se escolheu a ferramenta \textit{Render Monkey}.

Para poder finalmente verificar a viabilidade do tema em si, foi implementado um programa no qual é uma cena constituída por três esferas (com número de polígonos variável), em que cada uma faz uma movimentação diferente (em que garante-se que há oclusão e a distância em relação à câmera varia) e nas quais aplicam-se \textit{shaders} específicos.   

Assim, utilizando as ferramentas mencionadas anteriormente, foi possível coletar o número de quadros por segundo para diferentes números de polígonos. E dessa forma, gráficos (quadros por segundo x número de polígonos) para cada \textit{shader} implementado foram traçados (Anexo I), podendo então analisar experimentalmente suas complexidades algorítmicas.

\section{Procedimentos Futuros}

Assim, os próximos passos estão relacionados com a escolha de quais \textit{shaders} serão implementados e terão suas complexidades algorítmicas analisadas, como também com a modelagem de objetos tridimensionais possuindo diferentes números de polígonos e com a implementação do leitor desses objetos. 

Por fim, o método dos mínimos quadrados será utilizado para poder estimar o número de quadros por segundo de um \textit{shader}, dado um número n de polígonos, baseando-se na curva obtida experimentalmente pelos gráficos de cada \textit{shader} tanto no computador quanto no celular. 



