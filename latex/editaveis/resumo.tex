\begin{resumo}

 A utilização dos dispositivos móveis e da plataforma \textit{Android} tem crescido e constata-se a importância dos efeitos visuais em jogos e a sua limitação de desempenho. Assim, a proposta do trabalho se baseia no desenvolvimento de \textit{shaders} (programas responsáveis pelos efeitos visuais) para a plataforma \textit{Android} e para computador, em que suas complexidades algorítmicas serão analisadas, baseando-se na métrica de quadros por segundo. Além disso, o método dos mínimos quadrados será utilizado, para ajustar os valores obtidos a uma curva, podendo então, estimar qual a quantidade máxima de polígonos para se executar um programa na quantidade de quadros por segundo desejada para um determinado \textit{shader}. 

 \vspace{\onelineskip}
    
 \noindent
 \textbf{Palavras-chaves}: \textit{Android}, \textit{shaders}, dispositivos móveis, computação gráfica, jogos, complexidade algorítmica. 
\end{resumo}
