\chapter[Delimitação do Assunto]{Delimitação do Assunto}

O tema consiste no desenvolvimento de \textit{shaders} aplicados em objetos tridimensionais - com número de polígonos variante - no qual em seguida renderiza-se a cena e são coletadas medições quanto ao número de quadros por segundo. Desta forma, é possivel variar a quantidade de polígonos de um objeto e traçar um gráfico quantidade de polígonos \textit{versus} quadros por segundo utilizando um determinado \textit{shader}. E assim, analisa-se experimentalmente a complexidade algorítmica desses \textit{shaders}, para posteriormente poder aplicar o método dos mínimos quadrados (como explicado na seção 2 Referencial Teórico), a fim de estimar o número de quadros por segundo de um \textit{shader} específico dado um número n de polígonos, baseando-se na curva obtida experimentalmente pelos gráficos.  

Como visto na seção 5 Referencial Teórico, a complexidade algorítmica não depende das condições do ambiente de realização dos experimentos. Um algoritmo possui a mesma complexidade mesmo sendo implementado utilizando-se diferentes linguagens de programação, por exemplo. Assim, é possivel aplicar a proposta em diferentes contextos, como utilizando os \textit{shaders} em computador e em celulares. 

A fim de analisar se o tema também podia ser expandido para o contexto \textit{mobile} e verificar se é factível dentro do prazo estipulado, primeiramente desenvolveu-se um \textit{shader} no qual se utiliza a técnica \textit{Gouraud Shading} (como explicado na seção 2 Referencial Teórico) aplicado em um objeto tridimensional, o octaedro, usando a linguagem Java (padrão do \textit{Android}). O mesmo programa também foi feito para computador utilizando a linguagem C++.  Dessa forma, foi possível averiguar que o tema também poderia ser estendido e aplicado na plataforma \textit{Android} e evidenciou-se as principais diferenças entre a \textit{OpenGL ES} - utilizada para celulares - e a \textit{OpenGL} que é utilizada em computadores (discutido na seção 5 Resultados Alcançados). 
