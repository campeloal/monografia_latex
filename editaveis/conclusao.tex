\chapter[Conclusao]{Conclusão}\

	Por meio do trabalho realizado, foi possível concluir que o processo de renderização como um todo, calculado por meio do tempo de renderização da GPU, tendeu a apresentar como complexidade algorítmica um polinômio de segundo grau para qualquer \textit{shader} (variando somente os coeficientes da função).  Já o processo relacionado ao \textit{vertex shader}, por meio dos experimentos realizados, foi possível perceber que a complexidade algorítmica se comportou linearmente e o relacionado ao \textit{fragment shader} se aproximou de um polinômio de segundo grau, também independentemente do \textit{shader} utilizado. Assim, todos os \textit{shaders} possuem a mesma complexidade algorítmica, porém as equações de cada um possuem coeficientes diferentes, que podem determinar qual \textit{shader} tem melhor ou pior desempenho.

	Analisando a teoria do processo de renderização da \textit{OpenGL} este resultado é consistente, pois o programa do \textit{vertex shader} é utilizado para cada vértice (sendo de ordem linear).  O do \textit{fragment shader}, por sua vez, é de ordem do segundo grau, já que seu programa é usado para cada fragmento (sendo uma matriz). Além disso, ao analisar a documentação da GLSL\footnote{http://www.opengl.org/documentation/glsl/}, percebe-se que um fragmento (ou vértice) específico não tem conhecimento dos seus fragmentos (ou vértices) vizinhos, então não é possível que a complexidade cresça de um polinômio de segundo grau para um de terceiro, por exemplo. Porém este resultado não é tão óbvio, pois ao analisar o programa do \textit{vertex shader} do Código \ref{vertex_program}, por exemplo, há somente uma atribuição, induzindo o programador  a achar que a complexidade é constante. E por meio deste trabalho foi possível perceber que não é.

	\lstinputlisting[language=C, caption = {Exemplo de programa do \textit{vertex shader}}, label = {vertex_program}]{codigos/red_vs.txt} 
	
	Assim, como todos os \textit{shaders} (do mesmo tipo) apresentam a mesma complexidade algorítmica, uma forma de comparar o desempenho entre eles, para um mesmo \textit{device}, é através do processo realizado neste trabalho (explicado na Seção \ref{result}), que resulta no cálculo das funções de cada \textit{shader}. Esta comparação pode ser feita por meio da análise destas funções obtidas, comparando-se os seus coeficientes. Esta análise pode ser realizada com relação a todo o processo de renderização (utilizando a medida de tempo de renderização feita pela GPU) ou especificadamente ao \textit{vertex shader} ou \textit{fragment shader} -- como neste trabalho, em que foram utilizados as medidas específicas de instruções por segundo por vértice/fragmento).  Isto pode ser feito para comparar diferentes \textit{shaders} ou para saber o quanto um \textit{shader} foi otimizado (comparando-se o anterior e o atual). 

	Outra contribuição importante foi quanto à automatização da maior parte deste processo de análise da complexidade algorítimica, como a estrutura para aplicação dos \textit{shaders}, média das medições, plotagem, ajuste das curvas e cálculo das funções. Assim, tal procedimento pode ser reproduzido de forma mais rápida e confiável. 
