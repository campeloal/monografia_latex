\begin{resumo}

 A utilização dos dispositivos móveis e da plataforma \textit{Android} tem crescido e constata-se a importância dos efeitos visuais em jogos, bem como a sua limitação atual de desempenho dos processadores gráficos destas plataformas. Assim, a proposta do trabalho se baseia na implementação e aproximação experimental da complexidade assintótica dos \textit{shaders} (programas responsáveis pelos efeitos visuais) para a plataforma \textit{Android}. Suas complexidades algorítmicas serão analisadas, baseando-se nas métricas de número de instruções por segundo e tempo de renderização em função da variação do número de polígonos renderizados. Além disso, o método dos mínimos quadrados será utilizado para ajustar os valores obtidos a uma curva, permitindo determinar qual curva mais se aproxima da função original.

 \vspace{\onelineskip}
    
 \noindent
 \textbf{Palavras-chaves}: \textit{Android}, \textit{shaders}, dispositivos móveis, computação gráfica, jogos, complexidade algorítmica. 
\end{resumo}
