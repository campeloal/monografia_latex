\chapter[Introdução]{Introdução}
\label{introduc}

\section{Contextualização e Justificativa}

	Conforme \cite{graphicsprog}, os gráficos em jogos são um fator tão importante que podem determinar o seu sucesso ou fracasso. O aspecto visual é um dos pontos principais na hora da compra, juntamente com o \textit{gameplay} (maneira que o jogador interage com o jogo). Assim, os gráficos estão progredindo na direção próxima dos efeitos visuais dos filmes, porém o poder computacional para atingir tal meta ainda tem muito a evoluir.

	 Neste contexto, o desempenho gráfico é um fator chave para o desempenho total de um sistema, principalmente na área de jogos, que também possui outros pontos que consomem recursos, como inteligência artificial, \textit{networking}, áudio, detecção de eventos de entrada e resposta, física, entre outros. E isto faz com que o desenvolvimento de efeitos visuais mais complexos se tornem mais difíceis ainda.

	O recente crescimento do desempenho de dispositivos móveis tornou-os capazes de suportar aplicações cada vez mais complexas. Além disso, segundo \cite{teapot}, dispositivos como \textit{smartphones} e \textit{tablets} têm sido amplamente adotados, emergindo como uma das tecnologias mais rapidamente propagadas. Dentro deste contexto, mais de 700 milhões de aparelhos utilizando a plataforma \textit{iOS} já foram vendidos e a ativação diária de aparelhos na plataforma \textit{Android} é de aproximadamente 1,5 milhões, sendo uns dos sistemas operacionais para dispositivos móveis mais utilizados.

	 Porém, de acordo com \cite{x3d}, a renderização gráfica para dispositivos móveis ainda é um desafio devido a limitações, quando comparada a de um computador, como por exemplo, relacionadas a CPU (\textit{Central Processing Unit}), desempenho dos aceleradores gráficos e consumo de energia. Os autores \cite{teapot} mostram que estudos prévios evidenciam que os maiores consumidores de energia em um \textit{smartphone} são a  GPU (\textit{Graphics Processing Unit}) e a tela, sendo a GPU responsável por realizar o processo de renderização.

	Assim, é possível analisar o desempenho do processo de renderização feito pela GPU -- em que diferentes \textit{shaders} são aplicados (responsáveis pela criação dos efeitos visuais) -- por meio da complexidade algorítmica. Além disso, a análise do desempenho dos \textit{shaders} é uma área pouco explorada, tornando o tema abordado neste trabalho mais relevante ainda. 

	Então, o tema consiste no desenvolvimento de \textit{shaders} aplicados em objetos tridimensionais -- com número de polígonos variável -- os quais permitem a coleta de medições relacionadas ao desempenho de um dispositivo, com relação à renderização feita pela GPU. Desta forma, é possível variar a quantidade de polígonos de um objeto e traçar um gráfico quantidade de polígonos \textit{versus} métrica de desempenho, utilizando um determinado \textit{shader}. E assim, é possível identificar qual curva gerada pelo gráfico, e consequentemente, determinar a complexidade algorítmica do \textit{shader} para um determinado dispositivo.  

\section{Objetivos Gerais}

O objetivo geral do trabalho é a análise da complexidade algorítmica dos \textit{shaders} para diferentes dispositivos móveis, tanto para todo o processo de renderização quanto para somente parte dele (\textit{vertex} e \textit{fragment shaders}).

\section{Objetivos Específicos}

Os objetivos específicos do trabalho são:

\begin{itemize}
 \item Configurar os ambientes de desenvolvimento para a plataforma \textit{Android} e \textit{iOS};
 \item Implementar os \textit{shaders} para a plataforma \textit{Android} e \textit{iOS};
\item Procurar ferramentas de coleta da medição de desempenho para o \textit{device} utilizado;
\item Identificar qual métrica será utilizada para a análise de complexidade;
\item Coletar as medições estabelecidas;
\item Automatizar o cálculo dos ajustes das curvas e plotagem dos gráficos
\item Analisar a complexidade algorítmica dos \textit{shaders}, baseado-se nas curvas obtidas.
\end{itemize}

\section{Organização do Trabalho}

	No próximo capítulo serão apresentados os conceitos teóricos necessários para o entendimento do trabalho, como, por exemplo, definição de \textit{shaders} e sua função no processo de renderização, a biblioteca gráfica utilizada,  a fundamentação teórica matemática para implementação dos \textit{shaders}, complexidade algorítmica, método dos mínimos quadrados, entre outros. 

	No desenvolvimento, os passos tomados no trabalho são descritos, enfatizando como foi feita a configuração do ambiente, quais equipamentos foram utilizados, como foram feitas as implementações dos \textit{shaders} e a automatização dos cálculos e plotagem dos gráficos.

	Nos resultados alcançados são descritos os resultados obtidos através da análise da complexidade algorítmica dos \textit{shaders} implementados, seguido das conclusões do trabalho realizado, em que mostra-se as contribuições obtidas.   